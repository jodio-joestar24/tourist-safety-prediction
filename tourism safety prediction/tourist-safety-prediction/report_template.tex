\documentclass[a4paper,12pt]{article}
\usepackage{geometry}
\geometry{margin=1in}
\usepackage{graphicx}
\usepackage{amsmath}
\usepackage{parskip}
\usepackage{enumitem}
\usepackage[utf8]{inputenc}
\usepackage[T1]{fontenc}
\usepackage{times}

\begin{document}

\title{Tourism Safety Prediction: A Machine Learning Web Application}
\author{Your Name}
\date{October 2025}
\maketitle

\begin{abstract}
This project develops a web application to predict safety scores for tourism destinations using machine learning. Built with Flask, scikit-learn, and Pandas, it uses a Random Forest Regressor trained on a tourism dataset to provide safety predictions via a user-friendly interface. The system includes data visualization and downloadable results, designed for local execution.
\end{abstract}

\section{Introduction}
Tourism safety is critical for travelers. This project leverages machine learning to predict safety scores based on features like crime levels and political stability, using the dataset "combined_tourism_dataset_labeled.csv". The objectives are:
\begin{itemize}
    \item Develop a Random Forest model for safety score prediction.
    \item Create a Flask-based web interface for user inputs and results.
    \item Provide visualizations for dataset insights.
\end{itemize}

\section{System Analysis}
\subsection{Problem Statement}
Travelers need reliable safety predictions for destinations. Manual assessment is inefficient; this project automates it using ML.

\subsection{Proposed System}
A Flask web app with:
\begin{itemize}
    \item Input form for features (e.g., Place Type, Crime Level).
    \item Random Forest Regressor for predicting safety scores.
    \item Plotly-based dashboard for insights.
\end{itemize}

\section{System Design}
\subsection{Architecture}
\begin{itemize}
    \item Frontend: HTML, CSS, Bootstrap, Plotly.
    \item Backend: Flask, scikit-learn, Pandas.
    \item Data: CSV-based, no database.
\end{itemize}

\subsection{Data Flow Diagram}
% Add DFD using a tool like LaTeX TikZ or insert as image
% \includegraphics{dfd.png}

\section{Implementation}
\subsection{Technologies Used}
\begin{itemize}
    \item Python 3.8+, Flask, scikit-learn, Pandas, Plotly.
    \item Random Forest Regressor for prediction.
\end{itemize}

\subsection{Algorithm}
Random Forest Regressor:
\begin{enumerate}
    \item Preprocess categorical features using OrdinalEncoder.
    \item Train on tourism dataset to predict `score`.
    \item Save model and preprocessor for web app use.
\end{enumerate}

\section{Results}
The app allows users to input features and get safety scores. Example:
\begin{itemize}
    \item Input: Place Type = Beach, Crime Level = Low
    \item Output: Predicted Safety Score = 3.2
\end{itemize}
The dashboard shows average scores by Place Type.

\section{Conclusion}
The project successfully delivers a functional ML-based web app for tourism safety prediction, suitable for MCA final-year submission.

\section{Future Scope}
\begin{itemize}
    \item Integrate a database (e.g., SQLite) for storing predictions.
    \item Add more visualizations and advanced ML models.
    \item Deploy on cloud platforms like Heroku.
\end{itemize}

\end{document}